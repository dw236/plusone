\documentclass{article}

\usepackage{url}

\begin{document}

Chakraborty and Nagwani proposed a K-means algorithm that can update its clustering given new data~\cite{chakraborty2011analysis}.  They evaluated its performance~\cite{chakraborty2011kmeans} and compared it to another incremental clustering algorithm~\cite{chakraborty2011performance}.

Scully~\cite{sculley2010web} gives an algorithm for performing K-means on big datasets with ``minibatches'': that is, by only looking at a small piece of the data at once.  I discovered the paper via \url{http://metaoptimize.com/qa/questions/5578/incremental-k-means-clustering}.  At that page, there is also a link to a simple online algorithm at \url{http://www.cs.princeton.edu/courses/archive/fall08/cos436/Duda/C/sk_means.htm}.

\bibliographystyle{plain}
\bibliography{di}

\end{document}
