An early version of topic modeling is seen in Latent Semantic
Indexing (LSI) which views a document as a combination of a larger
set of topics.  Topics are in turn about certain words
(represented by positive numbers) or not about certain words
(represented by negative numbers) \cite{Papadimitriou1997}.  They
provided a linear algebra algorithm that learned the structure
of their model.  They also argued that this model provides
insight into the wide applicability of principal
components analysis in data analysis. The mathematical tool at the core
of their work (and PCA) was the singular value
decomposition.

Shortly thereafter, Puzicha and Hofmann \cite{Hofmann04} provided a
probabalistic framework, termed probabilisitc latent
semantic indexing (PLSI). Topics are modelled as probability
distributions over words (no negative numbers). Documents are
then assumed to be generated as a mixture of these
topics. 

The large number of parameters in the PLSI model motivated
Blei, Ng and Jordan to develop a generative model for these parameters,
which resulted in the enormously influential Latent Dirichlet
Allocation (LDA) topic model \cite{Blei2003a}.

The most closely related works in theory are the aforementioned
algorithms provided by Arora et.al. \cite{Arora2012} which learn PLSI
under certain assumptions, and a breakthrough by
Anandkumar~\cite{AnandLDA} which gives a provably convergent algorithm
for learning LDA with polynomial data requirements.  There is a long
literature in learning mixture models in statistics and
more recently in theoretical computer science. See
\cite{MoitraValiant} for a recent breakthrough on learning mixtures of
Gaussians and a discussion of this field.

There is ample work describing methods for optimizing
LDA \cite{BleiCACM}.  Recent examples are described
in \cite{McCallumMALLET}, \cite{BleiCBA}. 

There is also significant work on extending LDA and topic models to
better model data as well as to augment them with other types of
information.  These are discussed in \cite{BleiCACM}.  The
hierarchical LDA model \cite{BleiCM} is perhaps the most relevant.
This model is based on the chinese restuarant process where new
customers arrive and chose to join a table or create a new one.  This
process can be used to create a hierarchical structure of documents
and topics. The authors argue that this is a better model for real
data and we tend to agree though our preliminary experiments do not
bear this out.  Still, we began by understanding the simpler model
which continues to have wide applications.

Other examples of variations of the LDA model include adding the
notion of a manifold to the model~\cite{DTM}, adding partially labeled
data~\cite{Partial}, and sparse LDA models~\cite{MimnoSparse}. 

This paper, in addition to providing some insight into the
effectiveness of LDA algorithms, seeks to provide an infrastructure to
evaluate effective topic modeling methods. We should point out that
the machine learning community has made efforts in this regard.  See
for example MLComp \cite{MLComp}. We also mention the Java-based
infrastructure that we used for topic modelling \cite{McCallumMALLET}
which is part of the larger MALLET machine learning package.
