An early version of topic models could be seen in latent semantic
indexing which views a document as being about a subset of a larger
set of topics.  Topics are in turn are about (positive numbers) or not
about (negative numbers) certain words \cite{Papadimitriou1997}.  They
used this view to provide an algorithm based on linear algebra that
learned the structure of their model.  They also argued that this
model provided some insight into the wide applicability of principal
components analysis in data analysis. The algorithmic tool at the core
of their work (and principal components) was the singular value
decomposition.

Shortly thereafter, a probabilistic framework was provided by Puzicha
and Hofmann \cite{Hofmann04}.  They termed their method probabilisitc latent
semantic indexing (pLSI), and modelled topics as probability
distributions over words (no negative numbers in this description).
Documents were then presumed to be generated as a mixture of these
topics.  The large number of parameters in the PLSI model motivated
Blei, Ng and Jordan to provide a generative model for these parameters
as well; they provided the enormously influential Latent Dirichlet
Allocation(LDA) topic model \cite{Blei2003a}.

The most closely related work in theory are the aforementioned
algorithms provided by Arora et.al. \cite{Arora2012} which learn pLSI
under certain assumptions, and a breakthrough by
Anandkumar\cite{AnandLDA} which gives a provably converging algorithm
for learning LDA with polynomial data requirements.  There is a long
literature in learning mixtures of distributions in statistics and
more recently in theoretical computer science. See
\cite{MoitraValiant} for a recent breakthrough on learning mixtures of
Gaussians and a discussion of this field.

There is ample work describing methods for optimizing
LDA \cite{BleiCACM}.  Again, recent a recent examples are described
in \cite{MalletGibbs}, \cite{BleiCBA}. 

There is also significant work on extending LDA and topic models to
better model data as well as to augment them with other types of
information.  These are discussed in \cite{BleiCACM}.  The
hierarchical LDA model \cite{BleiCM} is perhaps the most relevant.
This model is based on the chinese restuarant process where new
customers arrive and chose to join a table or create a new one.  This
process can be used to create a hierarchical structure of documents
and topics. The authors argue that this is a better model for real
data and we tend to agree though our preliminary experiments do not
bear this out.  Still, we began by understanding the simpler model
which continues to have wide application.  

Other examples of variations of the LDA model include adding the
notion of a manifold to the model \cite{DTM}, adding partially labeled
data in \cite{}, and sparse LDA models
\cite{MimnoSparse}. 

This paper, in addition to providing some insight into the
effectiveness of LDA algorithms, seeks to provide an infrastructure to
evaluate effective topic modelling methods. We should point out that
the machine learning community is made efforts in this regard.  See
for example, MLComp \cite{MLComp}. We also highlight the java based
infrastructure that we used for topic modelling \cite{McCallumMALLET}
which is part of a larger MALLET java package for machine learning.
