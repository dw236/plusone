\documentclass{article}

\usepackage{amsmath}
\usepackage{amsthm}
\usepackage{mathtools}

\newcommand{\DefEq}{\stackrel{\mathrm{def}}{=}}

\DeclareMathOperator{\D}{d}
\DeclareMathOperator{\Range}{range}

\newtheorem{Problem}{Problem}

\begin{document}

TODO: Summary

\section{Problem Statement}

Fix a \(M \times k\) word-topic matrix \(A\).
Suppose there is a document with topic distribution \(t\).
We observe a vector \(w\) of word counts sampled from \(At\).
\begin{Problem}
    \label{Problem:InferTopics}
    Given the words \(w\), estimate the topic distribution \(t\).
\end{Problem}

\section{Algorithms}

We consider three algorithms to solve Problem~\ref{Problem:InferTopics}.

\subsection{\(\ell_2\) Projection}
\label{Sec:L2Alg}

Return the vector \(t\) which minimizes \(|w - A t|_2\).
In other words, orthogonally project \(w\) to \(w' \in \Range(A)\), and then make \(w' = A t\).

We can also think of this in terms of matrix inverses.
Let \(A^+\) be the Moore-Penrose pseudoinverse of \(A\): so \(A^+A = I_k\), and the Frobenius norm of \(|A^+|\) is as small as possible.
The \(\ell_2\) projection algorithm returns the topic vector \(A^+w\).

\subsection{Kleinberg-Sandler}
\label{Sec:KSAlg}

The Kleinberg-Sandler algorithm~\cite{kleinberg2004using} uses a different generalized inverse.
Let \(B\) be a matrix such that \(BA = I_k\).
If there are many such \(B\), pick one with the smallest maximum entry.
If you want to think in terms of \(\ell_1\) and \(\ell_\infty\), we are minimizing \(|B|_{1,\infty} = \max_{|x|_1 = 1} |B x|_\infty\).
Given a word vector \(w\), this algorithm returns \(Bw\).

\subsection{Maximum Likelihood}
\label{Sec:MLAlg}

Each word of a document with topic distribution \(t\) is drawn from the distribution \(A t\):
\[ \Pr[\text{next word is \(j\)} | t] = (A t)_j \ldotp \]
Given the word counts \(w\), we apply Bayes' rule:
\begin{align*}
    \Pr[t | w] \propto & \Pr[t] \Pr[w | t]
\\  = & \Pr[t] \prod_{j=1}^M (A t)_j^{w_j} \ldotp
\end{align*}
It is common to assume \(t\) is drawn from a Dirichlet distribution:
\begin{equation}
    \label{Eq:TopicProbWithAlpha}
    \Pr[t | w] \propto \prod_{i = 1}^k t_i^{\alpha - 1} \prod_{j=1}^M (A t)_j^{w_j} \ldotp
\end{equation}
Maximizing (\ref{Eq:TopicProb}) is NP hard when \(\alpha < 1\), but for \(\alpha \ge 1\), it can be done in polynomial time~\cite{conf/nips/SontagR11}.
For this analysis we will assume \(t\) is drawn from the uniform distribution (\(\alpha=1\)):
\begin{equation}
    \label{Eq:TopicProb}
    \Pr[t | w] \propto \prod_{j=1}^M (A t)_j^{w_j} \ldotp
\end{equation}
Our algorithm then maximizes (\ref{Eq:TopicProb}) exactly.

\section{Comparison: Disjoint Topics}

In this section, we assume no word has non-zero weight in more than one topic.
Given a vector of word counts \(w\), let \(n_i\) be the number of words in topic \(i\): \(n_i = \sum_{A_{ij} > 0} w_j\).

\subsection{Algorithm~\ref{Sec:MLAlg}}

The maximum-likelihood algorithm assigns topic \(i\) a weight proportional to \(n_i\), since (\ref{Eq:TopicProb}) becomes
\[ \Pr[t | w] \propto \prod_{i=1}^k t_i^{n_i} \]
which is maximized by \(t_i \propto n_i\).

\subsection{Algorithm~\ref{Sec:KSAlg}}

The Kleinberg-Sandler behaves in the same way as Algorithm~\ref{Sec:MLAlg} as the number of samples from \(A t\)  (document length) approaches infinity.

With less data, the behavior of the algorithm is underspecified.
The matrix \(B\) is chosen to minimize \(\max_{ji} |B_{ji}|\), but the minimum might not be unique.
One possibility is that the algorithm will behave the same way as Algorithm~\ref{Sec:MLAlg}: since \(A\) has the form
\[
    A =
    \begin{pmatrix}
        *      &        &
    \\  \vdots &        &
    \\  *      &        &
    \\         & *      &
    \\         & \vdots &
    \\         & *      &
    \\         &        & \vdots
    \end{pmatrix}
    ,
\]
    where \(*\) denotes a non-zero value and each column sums to \(1\), a possible generalized inverse is
\[
    B =
    \begin{pmatrix}
        1 & \cdots & 1 &   &        &   &
    \\    &        &   & 1 & \cdots & 1 &
    \\    &        &   &   &        &   & \cdots
    \end{pmatrix}
    ,
\]
    giving \((B w)_i \propto n_i\).
Indeed, the minimum value of \(\max_{ji} |B_{ji}|\) will always be \(1\), and because of the requirement that \(B A_i = e_i\) for any topic \(i\), we will always have \(B_{ji} = 1\) whenever \(A_{ij} > 0\).
On the other hand, there is some flexibility in the other entries of \(B\): for example, if
\[
    A =
    \begin{pmatrix}
        1/2 & 0
    \\  1/2 & 0
    \\  0   & 1
    \end{pmatrix}
    ,
\]
    then Algorithm~\ref{Sec:KSAlg} might choose
\(
    B =
    \begin{pmatrix}
         1 & 1 & 0
    \\  -1 & 1 & 1
    \end{pmatrix}
\)
instead of
\(
    B =
    \begin{pmatrix}
        1 & 1 & 0
    \\  0 & 0 & 1
    \end{pmatrix}
\).
With a very large number of samples, \(w_1\) and \(w_2\) will both be very close to equal and so it will not matter which matrix the algorithm chooses.

\subsection{Algorithm~\ref{Sec:L2Alg}}

The \(\ell_2\) projection algorithm behaves well when the topics are uniform distributions or when the number of samples is large, but it behaves strangely when the topics are non-uniform and the document is short.

\subsubsection{Uniform Topics}

Suppose the \(i\)-th topic is the uniform distribution over \(M_i\) words:
\[
    A =
    \begin{pmatrix}
        1/M_1  &        &        &
    \\  \vdots &        &        &
    \\  1/M_1  &        &        &
    \\         & 1/M_2  &        &
    \\         & \vdots &        &
    \\         & 1/M_2  &        &
    \\         &        & \vdots &
    \\         &        &        & 1/M_k
    \end{pmatrix}
\]
(TODO: diagram)

In this case, the \(\ell_2\) projection algorithm assigns topic \(i\) a weight proportional to the number of observed words from that topic.
This is the same behavior as Algorithm~\ref{Sec:MLAlg}, or Algorithm~\ref{Sec:KSAlg} when it decides to set the unspecified entries of \(B\) to zero:
\[
    A^+ = B =
    \begin{pmatrix}
        1 & \cdots & 1 &   &        &   &        &
    \\    &        &   & 1 & \cdots & 1 &        &
    \\    &        &   &   &        &   & \cdots &
    \\    &        &   &   &        &   &        & 1
    \end{pmatrix}
    \ldotp
\]

\subsubsection{Non-Uniform Topics}

Suppose
\[
    A =
    \begin{pmatrix}
        1 - \epsilon & 0
    \\  \epsilon     & 0
    \\  0            & 1
    \end{pmatrix}
    \ldotp
\]
Then the pseudoinverse is (approximately)
\[
    A^+ \approx
    \begin{pmatrix}
        1 + \epsilon & \epsilon & 0
    \\  0            & 0        & 1
    \end{pmatrix}
    \ldotp
\]
Consider two documents \(w = \begin{pmatrix} 1 \\ 0 \\ 1 \end{pmatrix}\) and  \(w' = \begin{pmatrix} 0 \\ 1 \\ 1 \end{pmatrix}\).
Then \(A^+ w \approx \begin{pmatrix} 1 + \epsilon \\ 1 \end{pmatrix}\), which is almost fair to both topics, but \(A^+ w' \approx \begin{pmatrix} \epsilon \\ 1 \end{pmatrix}\): the evidence of word 2 is ignored.

\section{Comparison: Overlapping Topics}

\subsection{Example}

\[
    A =
    \begin{pmatrix}
        1/2 & 0
    \\  1/2 & 1/2
    \\  0   & 1/2
    \end{pmatrix}
\]

Here, the three algorithms all do different things.

The maximum-likelihood algorithm predicts \(t_1 \propto w_1\) and \(t_2 \propto w_3\), and ignores the overlapping word \(w_2\), because (\ref{Eq:TopicProb}) becomes
\[ \Pr[t | w] \propto t_1^{w_1} t_2^{w_3} \]

The \(\ell_2\) projection and Kleinberg-Sandler algorithms, respectively, use the generalized inverses
\[
    A^+ =
    \begin{pmatrix}
         4/3 & 2/3 & -2/3
    \\  -2/3 & 2/3 &  4/3
    \end{pmatrix}
\]
    and
\[
    B =
    \begin{pmatrix}
         1 & 1 & -1
    \\  -1 & 1 &  1
    \end{pmatrix}
    \ldotp
\]

\subsection{Example}

\[
    A =
    \begin{pmatrix}
        1/2 & 0
    \\  1/2 & 1/2
    \\  0   & 1/2
    \end{pmatrix}
\]

Again, the three algorithms all do different things.

The \(\ell_2\) projection and Kleinberg-Sandler algorithms, respectively, use the generalized inverses
\[
    A^+ =
    \begin{pmatrix}
        10/7 & 1/7 & -2/7
    \\  -4/7 & 8/7 &  5/7
    \end{pmatrix}
\]
    and
\[
    B =
    \begin{pmatrix}
        6/5               & 3/5        & -6/5
    \\  -(1 + \delta) / 2 & 1 + \delta & 1 - 2 \delta
    \end{pmatrix}
    ,
\]
where the value of \(\delta \in [-1/10, 1/5]\) is undetermined.

The maximum-likelihood algorithm will choose the vector \(t\) which maximizes
\[ p \DefEq \Pr[ t | w ] \propto t_1^{w_1} (t_1 + 2 t_2)^{w_2} t_2^{w_3} \ldotp \]
Substituting \(t_2 = 1 - t_1\) and differentiating \(\log p\) with respect to \(t_1\), we have
\[ \frac {\D} {\D t_1} \log p = \frac {w_1} {t_1} - \frac {w_2} {2 - t_1} - \frac {w_3} {1 - t_1} \ldotp \]
The optimal \(t\) will either be an extreme \(t \in \big\{ \begin{psmallmatrix} 1 \\ 0 \end{psmallmatrix}, \begin{psmallmatrix} 0 \\ 1 \end{psmallmatrix} \big\}\) or satisfy
\[ t_1 (t_1 + 2 t_2) w_1 - t_1 t_2 w_2 - t_1 (t_1 + 2 t_2) w_3 = 0 \ldotp \]
To see a plot of \(t_1\) with respect to \(w_1\) and \(w_3\), run the maxima batch file \texttt{overlapping\_topics\_ml.wxm}.

\bibliography{bib}{}
\bibliographystyle{plain}

\end{document}
